\documentclass[journal]{IEEEtran}
\usepackage{blindtext}
\let\labelindent\relax
\usepackage[inline]{enumitem}
\usepackage{graphicx}
\usepackage[acronym]{glossaries}
\usepackage{subcaption}
\usepackage[bookmarksopen, bookmarksdepth=2, breaklinks=true]{hyperref}
\let\labelindent\relax

% *** GRAPHICS RELATED PACKAGES ***
%
\ifCLASSINFOpdf
\else
\fi


\newacronym{ble}{BLE}{Bluetooth Low Energy}

\hyphenation{op-tical net-works semi-conduc-tor}


\begin{document}
\title{SmartOrders: An Offline Smartphone Ordering System for Restaurants}

\author{\begin{center}
\begin{tabular}{c c} 
 Marius Becherer & Michael Zipperle \\ 
 \textit{259158} & \textit{259564} \\
 Marius.Becherer@hs-furtwangen.de & Michael.Zipperle@hs-furtwangen.de \\
\end{tabular}
\end{center}}%
        
%\author{Christian Laustsen, \textit{20176018},
%        Anders Rikvold, \textit{20176009},
%        and Michael Zipperle, \textit{20176059}}% <-this % stops a space
%\thanks{M. Shell is with the Department
%of Electrical and Computer Engineering, Georgia Institute of Technology, Atlanta,
%GA, 30332 USA e-mail: (see http://www.michaelshell.org/contact.html).}% <-this % stops a space
%\thanks{J. Doe and J. Doe are with Anonymous University.}% <-this % stops a space
%\thanks{Manuscript received April 19, 2005; revised January 11, 2007.}}

% The paper headers
\markboth{KAIST CS442 - Mobiles Computing and Applications, June 2017}%
{KAIST CS442 - Mobile Comsputing and Applications, June 2017}

% make the title area
\maketitle


\begin{abstract}
%\boldmath
Orderings systems for restaurants offer a great way to offload work from restaurants, in taking orders, and lessens the burden of customers needing to acquire and interact with a waiter. In this paper we investigate the feasibility of an offline ordering system for restaurants that only uses commodity hardware (smartphones) and communicates entirely over BLE. This stands in contrast to current systems, which either require special hardware at the tables or that the customer uses a smartphone application that functions over the Internet.

Using technology such as BLE and NFC we propose a system to handle the interaction between the restaurant and customer, along with a prototype implementation, \textit{SmartOrder}, that demonstrates the feasibility of the system. SmartOrder shows that common concerns such as the range and throughput of \gls{ble} are either not a concern or at least acceptable. The prototype implementation achieves a range of 71 meters with no obstacles and 12 meters around corners (6 meters on each side). Data throughput is shown to be somewhat slow, but this is acceptable since data transfers happen infrequently. Finally, both of these limitations are addressed with potential solutions.

\end{abstract}

% Note that keywords are not normally used for peerreview papers.
%\begin{IEEEkeywords}
%IEEEtran, journal, \LaTeX, paper, template.
%\end{IEEEkeywords}

% For peerreview papers, this IEEEtran command inserts a page break and
% creates the second title. It will be ignored for other modes.
\IEEEpeerreviewmaketitle


% *** START OF SECTIONS ***--------------------------------------------

\section{Introduction}
Imagine you have just arrived in a restaurant, and taken a seat at a table. In order to start eating, you first ask the waiter for the menu, which he brings to you. After thinking about what you would like, you call the waiter to your table again, and tell him what you would like to eat. The waiter writes down your order, and takes it back to the restaurant kitchen, where your meal is prepared. Then, hopefully after not waiting too long, the waiter brings you your meal. 

% *** END OF SECTIONS ***---------------------------------------------

% needed in second column of first page if using \IEEEpubid
%\IEEEpubidadjcol

% An example of a floating figure using the graphicx package.
% Note that \label must occur AFTER (or within) \caption.
% For figures, \caption should occur after the \includegraphics.
% Note that IEEEtran v1.7 and later has special internal code that
% is designed to preserve the operation of \label within \caption
% even when the captionsoff option is in effect. However, because
% of issues like this, it may be the safest practice to put all your
% \label just after \caption rather than within \caption{}.
%
% Reminder: the "draftcls" or "draftclsnofoot", not "draft", class
% option should be used if it is desired that the figures are to be
% displayed while in draft mode.
%
%\begin{figure}[!t]
%\centering
%\includegraphics[width=2.5in]{myfigure}
% where an .eps filename suffix will be assumed under latex, 
% and a .pdf suffix will be assumed for pdflatex; or what has been declared
% via \DeclareGraphicsExtensions.
%\caption{Simulation Results}
%\label{fig_sim}
%\end{figure}

% Note that IEEE typically puts floats only at the top, even when this
% results in a large percentage of a column being occupied by floats.


% An example of a double column floating figure using two subfigures.
% (The subfig.sty package must be loaded for this to work.)
% The subfigure \label commands are set within each subfloat command, the
% \label for the overall figure must come after \caption.
% \hfil must be used as a separator to get equal spacing.
% The subfigure.sty package works much the same way, except \subfigure is
% used instead of \subfloat.
%
%\begin{figure*}[!t]
%\centerline{\subfloat[Case I]\includegraphics[width=2.5in]{subfigcase1}%
%\label{fig_first_case}}
%\hfil
%\subfloat[Case II]{\includegraphics[width=2.5in]{subfigcase2}%
%\label{fig_second_case}}}
%\caption{Simulation results}
%\label{fig_sim}
%\end{figure*}
%
% Note that often IEEE papers with subfigures do not employ subfigure
% captions (using the optional argument to \subfloat), but instead will
% reference/describe all of them (a), (b), etc., within the main caption.


% An example of a floating table. Note that, for IEEE style tables, the 
% \caption command should come BEFORE the table. Table text will default to
% \footnotesize as IEEE normally uses this smaller font for tables.
% The \label must come after \caption as always.
%
%\begin{table}[!t]
%% increase table row spacing, adjust to taste
%\renewcommand{\arraystretch}{1.3}
% if using array.sty, it might be a good idea to tweak the value of
% \extrarowheight as needed to properly center the text within the cells
%\caption{An Example of a Table}
%\label{table_example}
%\centering
%% Some packages, such as MDW tools, offer better commands for making tables
%% than the plain LaTeX2e tabular which is used here.
%\begin{tabular}{|c||c|}
%\hline
%One & Two\\
%\hline
%Three & Four\\
%\hline
%\end{tabular}
%\end{table}


% Note that IEEE does not put floats in the very first column - or typically
% anywhere on the first page for that matter. Also, in-text middle ("here")
% positioning is not used. Most IEEE journals use top floats exclusively.
% Note that, LaTeX2e, unlike IEEE journals, places footnotes above bottom
% floats. This can be corrected via the \fnbelowfloat command of the
% stfloats package.









% if have a single appendix:
%\appendix[Proof of the Zonklar Equations]
% or
%\appendix  % for no appendix heading
% do not use \section anymore after \appendix, only \section*
% is possibly needed

% use appendices with more than one appendix
% then use \section to start each appendix
% you must declare a \section before using any
% \subsection or using \label (\appendices by itself
% starts a section numbered zero.)
%


% use section* for acknowledgement
%\section*{Acknowledgment}


%The authors would like to thank...


% Can use something like this to put references on a page
% by themselves when using endfloat and the captionsoff option.
\ifCLASSOPTIONcaptionsoff
  \newpage
\fi



% trigger a \newpage just before the given reference
% number - used to balance the columns on the last page
% adjust value as needed - may need to be readjusted if
% the document is modified later
%\IEEEtriggeratref{8}
% The "triggered" command can be changed if desired:
%\IEEEtriggercmd{\enlargethispage{-5in}}

% references section

% can use a bibliography generated by BibTeX as a .bbl file
% BibTeX documentation can be easily obtained at:
% http://www.ctan.org/tex-archive/biblio/bibtex/contrib/doc/
% The IEEEtran BibTeX style support page is at:
% http://www.michaelshell.org/tex/ieeetran/bibtex/
%\bibliographystyle{IEEEtran}
% argument is your BibTeX string definitions and bibliography database(s)
%\bibliography{IEEEabrv,../bib/paper}
%
% <OR> manually copy in the resultant .bbl file
% set second argument of \begin to the number of references
% (used to reserve space for the reference number labels box)

\begin{thebibliography}{1}
\bibitem{BLEEnergy}
Siekkinen M., et al. Wireless Communications and Networking Conference Workshops (WCNCW), 2012 IEEE.
"How Low Energy is Bluetooth Low Energy? Comparative Measurements with ZigBee/802.15.4"


\end{thebibliography}

% biography section
% 
% If you have an EPS/PDF photo (graphicx package needed) extra braces are
% needed around the contents of the optional argument to biography to prevent
% the LaTeX parser from getting confused when it sees the complicated
% \includegraphics command within an optional argument. (You could create
% your own custom macro containing the \includegraphics command to make things
% simpler here.)
%\begin{biography}[{\includegraphics[width=1in,height=1.25in,clip,keepaspectratio]{mshell}}]{Michael Shell}

% You can push biographies down or up by placing
% a \vfill before or after them. The appropriate
% use of \vfill depends on what kind of text is
% on the last page and whether or not the columns
% are being equalized.

%\vfill

% Can be used to pull up biographies so that the bottom of the last one
% is flush with the other column.
%\enlargethispage{-5in}



% that's all folks
\end{document}


