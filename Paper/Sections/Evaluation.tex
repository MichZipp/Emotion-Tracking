\section{Evaluation}
Der Workshop ist nach unserer Einschätzung bei den Teilnehmern gut angekommen. Die fünf Leitfragen, welche in Abschnitt II.B formuliert wurden, haben uns dabei geholfen einen Workshop zu erstellen, der nicht vom eigentlichen Thema abweicht. 
In der Feedbackrunde wurde der Aufbau des Workshops positiv aufgenommen. Durch die Mischung aus kurzen Vorträgen und längeren Gruppenarbeiten konnten wir die Teilnehmer für dieses Thema begeistern. Wir denken es war der richtig ein Fundament zu schaffen auf dem schließlich in Einzel- oder Gruppenarbeit aufgebaut werden konnte. 
Die ausgewählten didaktischen Methoden sind, nach unserer Einschätzung, an den richtigen Stellen zum Einsatz gekommen. Bei der zweiten und dritten Aufgabe konnten die Teilnehmer zielgerichtet arbeiten und sind zu guten Ergebnissen gekommen. Es wurden sogar noch weitere Methoden entdeckt, welche wir in unserer Recherche nicht gefunden hatten.  Die erste Aufgabe, die als Einstieg für die Motivation gelten sollte, entsprach nicht ganz unseren Erwartungen. Ziel war es zu zeigen, dass Benutzer verschieden sind. Durch die Einteilung in zwei Gruppen mit einem negativen und positiven Beispiel konnte ein Versuch durchgeführt werden. Dieser bestätigte zwar unser erwartetes Ergebnis, aber vermutlich konnten eine Gruppe nicht allzu sehr motiviert werden, da unser Negativbeispiel auch als solches gelten sollte. In Zukunft wäre eine Aufgabe, welche die komplette Gruppe motiviert, besser gewesen. 
Zusammengefasst kann gesagt werden, dass wir mit unserem Workshop sehr zufrieden sind. Wir haben festgestellt, dass die eigentliche Aufgabe in einer umfangreichen Recherche liegt, um das Thema richtig zu verstehen. Gelingt dies, lässt sich eine gute Struktur mit geeignetem Inhalt erstellen.
  