\subsection{Umsetzung: Anwendungsfälle}\label{Umsetzung_Anwendungsfaelle}
\subsubsection*{Theorethischer Inhalt}
Im letzten Kapitel wurden Methoden aufgezeigt, um die Emotionen eines Nutzer während dessen Interaktion mit einer Maschine zu tracken. In diesem Kapitel gilt es herauszufinden, wie das Wissen über die Emotionen eines Nutzers genutzt werden kann, um die Interaktion zwischen Mensch und Maschine zu verbessern. Dazu sollen Anwendungsfälle entwickelt werden, bei denen die Nutzung von Emotion Tracking ein Vorteil aufbringt. Für dieses Kapitel ist keine theoretisches Wissen nötig und es kann somit direkt zu einer interaktiven Aufgabe mit den Teilnehmern des Workshops übergegangen werden.
\newline
\subsubsection*{3. Aufgabe}
\begin{tabular}{c c }
	Zeit: 20 min & Gruppenarbeit (4 Personen)\\
\end{tabular}

Die Teilnehmer des Workshops wurden in drei Gruppen a vier Personen unterteilt, jeder Gruppe wurde eine der folgenden Emotion Tracking Methode zugewiesen:
\begin{itemize}
	\item Gesichtsausdruck
	\item Sprachinformation
	\item Hauttemperatur
\end{itemize}

Jede Gruppe soll für ihre Methode die folgenden Aufgaben bearbeiten:
\begin{enumerate}
	\item Recherchieren Sie nach bestehenden Anwendungsfällen, bei denen Emotion Tracking zur Verbesserung der \ac{HMI} eingesetzt wird.
	\item Überlegen Sie sich Anwendungsfälle, bei denen Emotion Tracking zur Verbesserung der \ac{HMI} eingesetzt werden könnte.
	\item Welche Vor- und Nachteile der Ihnen zugeteilten Methode kommen auf.
\end{enumerate}
\vspace{2mm}
\subsubsection*{Ergebnisse}\label{Umsetzung_Anwendungsfaelle_Ergebnisse}
Nach Bearbeitung der Aufgabe, wurden jede Gruppe aufgefordert ihre Ergebnisse mit Hilfe eines Plakats den anderen Workshop Teilnehmer zu präsentieren. Im folgenden werden die Ergebnisse der Gruppen genauer erläutert, dabei werden zuerst mögliche Anwendugsfälle und dann Vor- und Nachteile der jeweiligen Methode aufgelistet.\newline

\noindent Gesichtsausdruck:
\begin{itemize}
	\item Anwendungsfälle:
	\begin{itemize}
		\item Schmerzerkennung von Patienten im Krankenhaus
		\item Erkennung von kriminellen Machenschaften eines Menschen
		\item Hilfestellung für Blinde
		\item Lernhilfe für Autisten
	\end{itemize}
	\item Vorteile:
	\begin{itemize}
		\item In den meisten Maschinen befindet sich heutzutage eine Webcam, die für die Gesichtsausdruckerkennung genutzt werden kann. Somit ist in diesen Fällen keine zusätzliche Hardware nötig und das Emotion Tracking lässt sich einfach einsetzen.		 
	\end{itemize}
	\item Nachteile:
	\begin{itemize}
		\item Ein Mensch kann seinen Gesichtsausdruck einfach manipulieren, in diesen Fällen, könnte durch Emotion Tracking nicht die richtige Emotion des Menschen festgestellt werden.
	\end{itemize}
\end{itemize}
\vspace{2mm}
Sprachinformation: 
\begin{itemize}
	\item Anwendungsfälle:
	\begin{itemize}
		\item Alexa Bestellservice: Durch die Analyse der Sprachinformationen kann festgestellt werden, ob der Nutzer wirklich eine Bestellung aufgeben will. Zum Beispiel kann ein Nutzer als Witz sagen "Alexa, bestelle mir Klopapier", Alexa würde durch die Sprachinformationen erkennen, dass es ein Witz ist und antworten "Du hast doch noch genug Klopapier".
		\item Gemütszustand im Auto: Führt der Fahrer eines Autos ein Gespräch während der Fahrt, könnte dieses Gespräch analysiert werden. Kommen Emotionen wie Frust,Ärger oder Müdigkeit beim Fahrer auf, könnte dieser aufgefordert werden, eine Pause einzulegen.
		\item Notruferkennung: Es gibt Fälle, bei denen Notrufe aufgegeben, die nicht der Wahrheit entsprechen. Durch die Emotion der Person, die den Notruf absetzt, kann festgestellt werden, ob die Angaben des Notrufs der Wahrheit entsprechen. Falls bedenken bei Angaben aufkommen, können diese hinterfragt werden.
		\item  Verhör: Wie bei der Prüfung des Notrufs, könnten bei einem Verhör die Sprachinformationen genutzt werden, um festzustellen, ob eine Person lügt.
		\item Seelsorge: Soziale Roboter können bei der Seelsorge die Emotionen des Menschen einbeziehen und somit diesen besser fördern.
	\end{itemize}
	\item Vorteile:
	\begin{itemize}
		\item Diese Methode lässt sich meist ohne zusätzliche Hardware einsetzten, da die meisten Maschinen bereits ein Mikrofon integriert haben.		 
	\end{itemize}
	\item Nachteile:
	\begin{itemize}
		\item Auch die Sprachinformation lassen sich einfach manipulieren. Eine Mensch kann seine Tonlage, Sprechtempo usw. verändern, wodurch die Genauigkeit der Methode sinkt.
		\item Um die Sprachinformationen einer Person analysieren zu können, muss dessen Aussage aufgenommen werden. Somit erhält man nicht nur Information über die Sprache und somit der Emotion sondern auch über den Inhalt der Aussage. Somit würde dieses Methode eine Person ausspionieren.
	\end{itemize}
\end{itemize}
\vspace{2mm}
Hauttemperatur:
\begin{itemize}
	\item Anwendungsfälle:
	\begin{itemize}
		\item Schlafanalyse: Durch das messen der Hauttemperatur während des Schlafes einer Person, können die Emotion  und somit ein optimale Schlafzeit bestimmt werden.
		\item Stressanlyse: Es kann festgestellt werden, wenn eine Person besonders viel Stress hat. Demnach könnte das Interface einer Maschine angepasst werden (z.B. beruhigende Farben), um den Stress der Person zu verringern.
	\end{itemize}
	\item Vorteile:
	\begin{itemize}
		\item Hauttemperatur Sensoren sind billig.
		\item Messmethode sehr einfach.	 
		\item Die Hauttemperatur kann von einer Person nicht einfach manipuliert werden.
	\end{itemize}
	\item Nachteile:
	\begin{itemize}
		\item Meist ist zusätzliche Hardware nötig, die an der Haut einer Person angebracht werden muss.
		\item Verhalten der Hauttemperatur unter dem Einfluss einer Krankheit (z.B. Fiber) oder der Außentemperatur
	\end{itemize}
\end{itemize}

Wie die Ergebnisse der Gruppenarbeit zeigen, ist es nicht einfach die Emotion einer Person in einem Anwendungsfall einzusetzen. Oft limitieren die Methoden dies, da diese zu ungenau, manipulierbar oder zusätzliche Hardware benötigt. Es zeigte sich auch, dass die erarbeiteten Anwendungsfälle der Gruppenarbeit sich auf bestimmte Gebiete reduzieren lassen. Zum einen gibt es Anwendungsfälle, bei denen durch Emotion Tracking die Wahrheit einer Tätigkeit bestimmt werden kann. Zum anderen können Gesundheitsschädliche Emotionen einer Person reduziert oder sogar vermieden werden. 

