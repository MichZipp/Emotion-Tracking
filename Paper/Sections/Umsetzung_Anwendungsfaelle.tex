\section{Umsetzung: Andwendungsfälle}\label{Umsetzung_Anwendungsfaelle}
Im letzten Kapitel wurden Methoden aufgezeigt, um die Emotionen eines Nutzer während dessen Interaktion mit einer Maschine zu tracken. In diesem Kapitel gilt es herauszufinden, wie das Wissen über die Emotionen eines Nutzers genutzt werden kann, um die Interaktion zwischen Mensch und Maschine zu verbessern. Dazu sollen Anwendungsfälle aufgezeigt werden, bei denen die Nutzung von Emotion Tracking ein Vorteil aufbringt. Für dieses Kapitel ist keine theoretisches Wissen nötig und es kann somit direkt zu einer interaktiven Aufgabe mit den Teilnehmern des Workshops übergegangen werden.

\subsection{Workshop Aufgabe}
Insgesamt waren es drei Gruppen a vier Personen, jeder Gruppe wurde eine der folgenden Emotion Tracking Methode zugewiesen:

\begin{itemize}
	\item Gesichtsausdruck
	\item Sprachinformation
	\item Hauttemperatur
\end{itemize}
Die Gruppen wurden aufgefordert, folgende Aufgabe durchzuführen.
\begin{itemize}
	\item Gruppengröße: 4 Personen
	\item Bearbeitungszeit: 20 Minuten
	\item Arbeitsverfahren: Recherche & Entwicklung
	\item Beschreibung: Die folgenden Aufgaben sind in Bezug zu einer bestimmten Emotion Tracking Methode zu bearbeiten:
	\begin{enumerate}
		\item Recherchieren Sie nach bestehenden Anwendungsfällen, bei denen Emotion Tracking zur Verbesserung der HMI eingesetzt wird.
		\item Überlegen Sie sich Anwendungsfälle, bei denen Emotion Tracking zur Verbesserung der HMI eingesetzt werden könnte.
		\item Welche Vor- und Nachteile der Ihnen zugeteilten Methode kommen auf.
	\end{enumerate}
\end{itemize}

Nach Bearbeitung der Aufgabe, wurden jede Gruppe aufgefordert ihre Ergebnisse mit Hilfe des Plakats den anderen Workshop Teilnehmer zu präsentieren.

\subsection{Workshop Ergebnisse}\label{Umsetzung_Anwendungsfaelle_Ergebnisse}