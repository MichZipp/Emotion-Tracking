\subsection{Diskussion}
Am Ende des Workshops wurden mit den Teilnehmern eine offene Diskussion durchgeführt. Im Folgenden werden die einzelnen Diskussionspunkte aufgelistet und die Ergebnisse beschrieben.
\vspace{2mm}
\vspace{2mm}

\textit{Kombination verschiedener Emotionen Tracking Methoden} 
\newline
In Kapitel \ref{Umsetzung_Anwendungsfaelle} zeigten sich Vor- und Nachteile ausgewählter Methoden zum Emotion Tracking. Die Teilnehmer waren sich einig, dass durch die Kombination verschiedener Methoden (beispielsweise Gesichtsausdruck mit Hauttemperatur) die Nachteile eliminiert werden können. Dies steigert die Genauigkeit und verringert die Manipulierbarkeit.
\vspace{2mm}

\textit{Datenschutz - Privatsphäre} \newline
Hierbei muss sich ein Nutzer die Frage stellen, will ich, dass die Maschine mit der ich interagiere meine Emotionen weiß? Emotion sind sehr sensible Daten und nach einer kurzen Umfrage, wäre kein Teilnehmer des Workshops damit einverstanden, dass bei deren Interaktion mit einer Maschine ihre Emotionen getrackt würden. Ebenso stellt sich die Frage, was macht eine Maschine mit den Emotionen eines Nutzers? - Passt es wirklich die Interaktion für den Nutzer an oder sammelt es auch die Daten und verkauft diese an Dritte weiter? Umfangreiche Interaktion mit einer Maschine findet heutzutage über ein mobiles Endgerät statt. Dabei hat das mobile Endgerät meist zu wenig Ressourcen, um die Emotion des Nutzers zu bestimmen. Somit lässt sich das mobile Endgerät nur als Eingabegerät nutzen und die Auswertung findet in der Cloud statt, was wiederum eine Gefahr für die Privatsphäre für Nutzer darstellt.
\vspace{2mm}
\vspace{2mm}

\textit{Umsetzungs-Nutzen Faktor}\newline
Kapitel \ref{Umsetzung_Anwendungsfaelle_Ergebnisse} zeigt, dass es eine große Herausforderung ist, die Emotionen eines Nutzers einzusetzen, um die Interaktion zwischen Mensch und Maschine zu verbessern. Nutzer sind unterschiedlich und reagieren somit unterschiedlich auf Änderungen des Interfaces. Somit stellt sich die Frage, lohnt sich bei der Entwicklung einer Schnittstelle zwischen Mensch und Maschine die Einbeziehung der Emotionen eines Nutzers? Da das Tracken der Emotionen relativ aufwendig ist, kamen die Teilnehmer des Workshops zu dem Schluss, dass der Einsatz nur in bestimmten Anwendungsfällen sinnvoll ist. Diese Anwendungsfälle beschränken sich größtenteils auf das Gesundheitswesen. Dabei kann Menschen mit Behinderung oder Senioren, die sich einsam fühlen, ein Assistenzsystem zur Verfügung gestellt werden, das individuell auf deren Emotionen eingehen kann. Für alltägliche Anwendungen, wie der Besuch von verschiedene Webseiten, ist der Einsatz zu aufwändig.

