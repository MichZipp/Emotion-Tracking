\section{Planung}


\subsection{Related Research / Recherche }

Nach der Bearbeitung des Artikels, konnte eine grobe Übersicht über das Thema gewonnen werden. Es stellt sich allerdings heraus, dass der Artikel kein Inhalt für einen 75 minütigen Workshop bot. Somit musst die Recherche erweitert werden, um ein höheres Abstraktionsniveau zu gewinnen. Zunächst untersuchten wir die zitierten Referenzen nach der Schneeballmethode. Hiermit konnte der Begriff Affective Computing\cite{Picard} nochmals verinnerlicht werden. Die genau Begriffsdefinition und die weiteren genannten Meilensteine, um das Ziel Affective Computing zu erreichen, sind sehr wichtig für das Verständnis der impliziten und expliziten Kanäle\cite{KimNS}. Für die verschiedene Methoden des Emotion Tracking sind ebenfalls Quellen angegeben\cite{cowie}\cite{zeng}\cite{lee}. Allerdings wurde die Recherche hier erweitert, sodass noch weitere Methoden, wie \cite{AutoEmotion} \cite{Carlos}. Im Bereich der Anwendungsfälle wird Affective Computing teilweise in der medizinischen Informatik\cite{medicine} und Sprachassisten\cite{sprachassi} angewendet. Eine beispielhafte Website, welche das Benutzerverhalten transparent überwacht und den Inhalt passenden dafür erzeugt, haben wir nicht gefunden. Es gibt jedoch einige Unterstützungen und Tools, wie Affectiva Emotion SDK\cite{affectiva} oder Microsoft Azure Cognitive Services\cite{MicrosoftAzure}, welche Affective Computing in einer Anwendung umsetzen soll.


\subsection{Agendaentwurf und Zeitplung}
Aufgrund dieser Recherche konnten wir ein Überblick über das Thema Affective Computing schaffen. Wir überlegten uns die Kernpunkte unseres Workshops und erstellten folgende fünf Fragen, welche die Teilnehmer nach dem Workshop beantworten können sollten:

\begin{itemize}
	\item Was sind Emotionen?
	\item Was ist Affective Computing?
	\item Welche Methoden gib es?
	\item Wie können diese Methoden in Anwendungen integriert werden?
	\item Welche Anwendungen setzen dies bereits um? 
\end{itemize}


Mit diesem Fragen wollten wir eine geeignete Agenda erstellen, welche vom Allgemeinen in die konkrete Methodik und Anwendung übergeht. Die Agenda bestand aus folgender Struktur, bei der die einzelnen Punkte mit einer ungefähren Zeit abgeschätzt wurde:

\begin{center}
	
\begin{tabular}{r l  l}
	1.&Motivation	&(15 min) \\
	2.&Ziel		&	(5 min) \\
	3.&Methoden	&	(30 min) \\
	4.&Umsetzung	&	(20 min) \\ 
	5.&Fazit		&	(10 min) \\
	6.&Diskussion	&(5 min) \\
\end{tabular}

\end{center}

Diese Agenda diente als Grundlage für die erste Besprechung. Bei einzelnen Punkten waren Änderungen notwendig, aber an dem Grundgerüst des Workshops gab es wenig zu bemängeln. Bevor wir nun den Workshop erstellten, haben wir uns nochmals in die Punkte der Agenda vertieft. 


\subsection{Methodik}  

Als Vorgabe für das didaktische Konzept wurde bereits der Workshop festgelegt, dessen Gestaltung war jedoch frei wählbar. Es stellte sich die Herausforderung entsprechende Methoden einzusetzen, welche sich positiv auf das Lernverhalten und die Motivation auswirken. Der Workshop sollte sich aus einem Vortrag zusammensetzen, welcher mit Gruppenaufgaben ergänzt wurde. Dabei soll der Vortrag Themen zusammenfassen und eine fundierte Basis liefern. In den Aufgaben soll dann das Wissen erweitert und ergänzt werden. Eingangs soll eine  Aufgabe in Einzelarbeit als Versuch durchgeführt werden, sodass die Teilnehmen für das Thema motiviert sind. Nach weiteren einführenden Informationen finden zwei Gruppenarbeiten statt, welche von den Workshopleitern betreut wird. Die Ergebnisse werden moderiert, sodass die Teilnehmer im Fokus stehen und eine erhöhte Aufmerksamkeit auf die Ergebnisse gegeben sein soll. Abschließend soll noch eine kurze Diskussion stattfinden, wobei die Teilnehmer ihr gesammeltes Wissen aus dem Workshop für die Diskussion transferieren können. Als Medien werden Beamer und Flipchart verwendet. Der Beamer soll die einzelnen Vortragsabschnitte unterstützen und die gesammelten Ergebnisse live präsentieren. Die Flipcharts können von der Gruppen zur Ergebnispräsentation verwendet werden. 



