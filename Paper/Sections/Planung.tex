\section{Planung}
\subsection{Recherche}
Nach der Bearbeitung des ausgewählten Artikels, konnte eine grobe Übersicht über das Thema gewonnen werden. Es stellt sich allerdings heraus, dass der Artikel kein Inhalt für einen 75 minütigen Workshop bot. Somit musste das Thema erweitert werden, um ein höheres Abstraktionsniveau zu erreichen. Zunächst untersuchten wir die zitierten Referenzen nach der Schneeballmethode. Hiermit konnte der Begriff Affective Computing nochmals verinnerlicht werden\cite{Picard}. Die genaue Begriffsdefinition und die weiteren genannten Meilensteine, um das Ziel Affective Computing zu erreichen, sind sehr wichtig für das Verständnis der impliziten und expliziten Kanäle\cite{KimNS}. 

In dem ausgewählten Artikel werden die Emotion Tracking Methoden Hauttemperatur und Hautwiderstand beschrieben \cite{cowie}\cite{zeng}\cite{lee}. Allerdings wurde das Thema hier erweitert, weitere Methoden wurden gefunden, welche in Kapitel \ref{MethodenEmotionTracking} erläutert werden. 

Im Bereich der Anwendungsfälle wird Affective Computing teilweise in der medizinischen Informatik und in Sprachassisten angewendet \cite{medicine}\cite{sprachassi}. Außerdem gibt es Tools wie Affectiva Emotion SDK \cite{affectiva} oder Microsoft Azure Cognitive Services \cite{MicrosoftAzure}, die es ermöglichen Affective Computing in eine Anwendung zu integrieren.

\subsection{Agendaentwurf und Zeitplanung}
Aufgrund dieser Recherche konnten wir uns einen Überblick über das Thema Affective Computing schaffen. Wir überlegten uns die Kernpunkte unseres Workshops und erstellten folgende fünf Fragen, welche die Teilnehmer nach dem Workshop sollten beantworten können:

\begin{itemize}
	\item Was sind Emotionen?
	\item Was ist Affective Computing?
	\item Welche Methoden zum Emotion Tracking gib es?
	\item Wie können diese Methoden in Anwendungen integriert werden?
	\item Welche Anwendungen setzen dies bereits um? 
\end{itemize}

Mit diesen Fragen wollten wir eine geeignete Agenda erstellen, welche vom Allgemeinen in die konkrete Methodik und Anwendung übergeht. Die Agenda bestand aus folgender Struktur, bei der die einzelnen Punkte mit einer gewissen Dauer abgeschätzt wurden:

\begin{center}
	\begin{tabular}{r l  l}
		1.&Motivation	&(15 min) \\
		2.&Ziel		&	(5 min) \\
		3.&Methoden	&	(30 min) \\
		4.&Umsetzung	&	(20 min) \\ 
		5.&Fazit		&	(10 min) \\
		6.&Diskussion	&(5 min) \\
	\end{tabular}
\end{center}

Diese Agenda diente als Grundlage für die erste Besprechung mit unserem Betreuer Prof. Dr. Stefan Betermieux. Bei einzelnen Punkten waren Änderungen notwendig aber das Grundgerüst für den Workshop war akzeptiert. Bevor wir nun den Workshop erstellten, haben wir uns nochmals in die Agendapunkte vertieft. 


\subsection{Methodik}  
Als Vorgabe für das didaktische Konzept wurde bereits ein interaktiver Workshop festgelegt. Es stellt sich die Herausforderung dar, entsprechende Methoden einzusetzen, welche sich positiv auf das Lernverhalten und die Motivation der Teilnehmer auswirken. Der Workshop soll sich aus einem Vortrag von theoretischen Inhalten und praktischen Aufgaben zu Anwendungen dieser Inhalte zusammensetzen. Eingangs soll eine  Aufgabe in Einzelarbeit als Versuch durchgeführt werden, sodass die Teilnehmer für das Thema motiviert sind. Nach weiteren einführenden Inhalten finden zwei Gruppenarbeiten statt, welche von den Workshopleitern betreut werden. Die Gruppenergebnisse werden moderiert, sodass die Teilnehmer im Fokus stehen und eine erhöhte Aufmerksamkeit für ihre erarbeiteten Ergebnisse erhalten. Abschließend soll noch eine kurze Diskussion stattfinden, wobei die Teilnehmer ihr gesammeltes Wissen aus dem Workshop für die Diskussion transferieren können. Als Medien werden Beamer und Flipchart verwendet. Der Beamer soll die einzelnen Vortragsabschnitte unterstützen und die gesammelten Ergebnisse live präsentieren. Die Flipcharts können von den Gruppen zur Ergebnispräsentation verwendet werden. Des Weiteren wurden Google Formulare eingesetzt, um die Ergebnisse einer Aufgabe schnell zu sammeln und anschließend grafisch darzustellen.



