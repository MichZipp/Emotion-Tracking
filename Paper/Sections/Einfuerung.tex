\section{Einführung}

Im heutigen Zeitalter muss nicht mehr allzu viel von Hand erledigt werden, denn für viele Anwendungen stehen Maschinen und sonstige Hilfsmittel bereit. Wir Menschen benutzen diese Geräte gerne, um uns den Alltag zu erleichtern. In den Computerwissenschaften hat sich ein eigener Fachbereich gebildet, welcher sich mit \ac{HMI} beschäftigt. Doch was passiert, wenn auf die Eingabe nicht die gewollte Ausgabe stattfindet? Wir Menschen kennen Missverständnisse in der Kommunikation zwischen Menschen. Oftmals ist eine Verständigung möglich, denn sollte die Sprache nicht ausreichen, kann die Gestik weiterhelfen, um einen ungenaue Nachricht zu übermitteln. Maschinen verstehen keine unserer alltäglichen Kommunikationshilfsmittel, sondern nur die vom Benutzer getätigte Eingabe. Sollte die Antwort nicht den Wünschen des Nutzers entsprechen, kann dabei Frust beim Menschen entstehen. 


Bei der Recherche nach einem geeigneten Thema für den Workshop, sind wir im Bereich Emotion Tracking und Affective Computing fündig geworden. Die Vision eine Kommunikation von Mensch zu Maschine an die Kommunikation von Mensch zu Mensch anzugleichen, überzeugte uns von dem Themengebiet, sodass wir ein Paper über '"A Suggestion to Improve User-Friendliness Based on Monitoring Computer User’s Emotions'" ausgewählt haben. Der Inhalt des Paper gliedert sich zum einen in einige grundlegenden Überlegungen zu Affective Computing und zum anderen in eine Emotion-Tracking-Methode. Wir haben uns dazu entschieden das Arbeitsfeld Emotion Tracking mit den Teilnehmern aufzuarbeiten.


Im nachfolgenden Bericht wird der Workshop genauer erläutert. Zunächst steht die Planung im Fokus. Hierbei möchten wir auf die Recherche eingehen und die Ergebnisse präsentieren. Hieraus konnte die Agenda entwickelt werden, welche den groben Rahmen des Workshop bildete. Natürlich sind die angewandten Methoden beim Workshop relevant, denn schließlich sollen die Teilnehmer für das Thema begeistert und motiviert werden.
Diese Überlegungen schließen das Kapitel Planung ab. Die Durchführung beschreibt den Workshopablauf, bei dem die theoretischen Inhalte und die Ergebnisse der Aufgaben zusammengefasst sind. In der Evaluation wird ein Rückblick auf die Vorbereitung und Durchführung geben. Dabei findet eine Evaluation statt, welches die Vorbereitung, den Aufbau, Inhalt und die angewendeten Methoden untersucht. 
