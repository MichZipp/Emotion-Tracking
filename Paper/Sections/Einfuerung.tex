\section{Einführung}

Im heutigen Zeitalter muss nicht mehr allzu viel von Hand erledigt werden. Für viele Anwendungen stehen Maschinen und sonstige Hilfsmittel bereit. Wir Menschen benutzen diese Geräte gerne, um uns den Alltag zu erleichtern. In den Computerwissenschaften hat sich hier sogar ein eigener Fachbereich gebildet, welcher sich mit HMI beschäftigt. Doch was passiert, wenn auf die Eingabe nicht die gewollte Ausgabe stattfindet? Wir Menschen kennen Missverständnisse in der Kommunikatkon zwischen Menschen. Oftmals ist Verständigugn möglich, denn sollte die Sprache nicht ausreichen, kann die Gestik weiterhelfen und einen ungenaue Nachricht übermitteln. Maschine verstehen keine Gestik und somit wird auch keine Information übermittelt. Dabei kann bei Benutzern Frust entstehen. 


Bei der Recherche nach einem geeigneten Thema, sind wir im Bereich Emotion Tracking und Affective Computing fündig geworden. Die Vision eine Kommunikation von Mensch zu Maschine an die Kommunikation von Mensch zu Mensch anzugleichen, überzeugte uns von dem Themengebiet, sodass ein Paper über "Name des Papers" ausgewählt haben. Der Inhalt des Paper gliedert sich zum einen in einige Grundlegenden Überlegungen zu Affective Computing und zum anderen in eine Emotion-Tracking-Methode. Wie haben uns dazu entschieden das Arbeitsfeld Emotion Tracking mit den Teilnehmern aufzuarbeiten.



- Ziele -





In der nachfolgenden Dokumentation wird der Workshop genauer erläutert. Zunächst steht die Planung im Fokus. Hierbei möchten wir auf die Recherche eingehen und die Ergebnisse präsentieren. Hieraus konnte die Agenda entwickelt werden, welche den groben Rahmen des Workshop bildete. Natürlich sind die angewandten Methoden und die richtige Didaktik beim Workshop relevant, denn schließlich sollen die Teilnehmer für das Thema begeistert und motiviert werden.
Diese Überlegungen schließen das Kapitel Planung ab. Die Durchführung beschreibt den Workshopablauf, bei dem die theoretischen Inhalte und die Ergebnisse der Aufgaben zusammengefasst sind. In der Evaluation werden die Ergebnisse diskutiert und Rückschlüsse auf den Workshop als auf das Themengebiet "Affective Computing" gezogen. Den Abschluss bildet ein kurzes Fazit, welches die Meinungen und Gedanken des Vorbereitungsteam wiedergibt. 
