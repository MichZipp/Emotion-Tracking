\section{Einführung}

Im heutigen Zeitalter kommen bei der Arbeit oftmals Maschinen zum Einsatz. Wir Menschen benutzen diese Geräte, um uns den Alltag zu erleichtern. Bei der Kommunikation zwischen Menschen können Missverständnisse auftreten. In den meisten Fällen können diese anhand der Gestik und Mimik des Kommunikationspartners festgestellt und darauf entsprechend reagiert werden. Kommunizieren wir mit einer herkömmlichen Maschine, wie einem Computer, steht für die Eingabe meist nur Maus und Tastatur zur Verfügung. Dies bedeutet, dass der Computer nicht die Möglichkeit hat Missverständnisse anhand menschlicher Gestik und Mimik zu verstehen. Diese Missverständnisse führen oft zu Frust bei Menschen. In den Computerwissenschaften hat sich ein eigener Fachbereich gebildet, der sich mit der \ac{HMI} beschäftigt. Teilgebiete dieses Bereichs erforschen Methoden zum Feststellen des emotionalen Zustandes eines Menschen während dessen Interaktion mit einer Maschine. Somit können Missverständnisse erkannt und beseitigt werden.

Dieser Artikel wurde im Rahmen des Moduls Ergonomie im Studiengang Mobile Systeme an der Fachhochschule Furtwangen erstellt. Die Aufgabe war es einen 75-minütigen Workshop mit einer Dokumentation zu erstellen. Das Thema durfte frei im Bereich Ergonomie ausgewählt werden. Bei der Recherche zu einem geeigneten Thema, sind wir im Bereich Emotion Tracking und Affective Computing fündig geworden. Die Vision eine Kommunikation von Mensch zu Maschine an die Kommunikation von Mensch zu Mensch anzugleichen, überzeugte uns von dem Themengebiet, sodass wir den Artikel "A Suggestion to Improve User-Friendliness Based on Monitoring Computer User’s Emotions" ausgewählt haben \cite{EmotionTrackingGSR}. Der Inhalt des Artikels gliedert sich in grundlegenden Überlegungen zu Affective Computing und stellt eine Kombination von Emotion-Tracking-Methoden vor.

In den nachfolgenden Kapiteln wird der Workshop genauer erläutert. Zunächst steht die Planung im Fokus. Hierbei wird die Recherche und deren Ergebnisse beschrieben, woraus die Agenda des Workshops entstand. Danach werden didaktische Methoden erarbeitet, um den Teilnehmern einen möglichst interessanten Workshop zu bieten und diese für das Thema zu motivieren. Das Kapitel Durchführung beschreibt den Workshopablauf, bei dem die theoretischen Inhalte und die Ergebnisse der Aufgaben dargestellt werden. In der anschließenden Evaluation wird ein Rückblick auf die Vorbereitung und Durchführung des Workshops gegeben. Dabei wird die Vorbereitung, der Inhalt und die didaktischen Methoden kritisch beurteilt.
