\section{Einführung}
Im heutigen Zeitalter muss nicht mehr allzu viel von Hand erledigt werden, denn für viele Anwendungen stehen Maschinen und sonstige Hilfsmittel bereit. Wir Menschen benutzen diese Geräte gerne, um uns den Alltag zu erleichtern. Bei der Kommunikation zwischen Menschen können Missverständnisse auftreten. In den meisten Fällen können diese anhand der Gestik und Mimik des Kommunikationspartners festgestellt werden und es kann entsprechend reagiert werden. Kommunizieren wir mit einer herkömmlichen Maschine, wie einem Computer, steht für die Eingabe meist nur eine Maus und Tastatur zur Verfügung. Dies bedeutet der Computer hat nicht die Möglichkeit, Missverständisse bei uns Menschen anhand unserer Gestik und Mimik, zu verstehen. Diese Missverständnisse führen oft zu Frust bei Menschen. In den Computerwissenschaften hat sich ein eigener Fachbereich gebildet, welcher sich mit der \ac{HMI} beschäftigt. Teilgebiete dieses Bereichs erforschen Methoden zum Feststellen des emotionalen Zustandes eines Menschen. Während seiner Interaktion mit einer Maschine, wodurch solche Missverständnisse erkannt und beseitigt werden können.

Dieser Artikel wurde im Rahmen des Faches Ergonomie im Studiengang Mobile Systeme an der Fachhochschule Furtwangen erstellt. Die Aufgabe war es einen 75-minütigen Workshop mit einer Dokumentation zu erstellen. Das Thema durfte frei im Bereich Ergonomie ausgewählt werden. Bei der Recherche nach einem geeigneten Thema, sind wir im Bereich Emotion Tracking und Affective Computing fündig geworden. Die Vision eine Kommunikation von Mensch zu Maschine an die Kommunikation von Mensch zu Mensch anzugleichen, überzeugte uns von dem Themengebiet, sodass wir den Artikel '"A Suggestion to Improve User-Friendliness Based on Monitoring Computer User’s Emotions'" ausgewählt haben \cite{EmotionTrackingGSR}. Der Inhalt des Artikels gliedert sich in grundlegenden Überlegungen zu Affective Computing und stellt eine Kombination von Emotion-Tracking-Methoden vor.

In den nachfolgenden Kapitel wird der Workshop genauer erläutert. Zunächst steht die Planung im Fokus. Hierbei wird die Recherche und deren Ergebnisse beschrieben, woraus die Agenda des Workshops entstand. Danach werden didaktische Methoden erarbeitet, um den Teilnehmern einen möglichst interessanten Workshop zu bieten und diese für das Thema zu motivieren. Das Kapitel Durchführung beschreibt den Workshopablauf, bei dem die theoretischen Inhalte und die Ergebnisse der Aufgaben dargestellt werden. In der anschließenden Evaluation wird ein Rückblick auf die Vorbereitung und Durchführung des Workshops gegeben. Dabei wird die Vorbereitung, der Inhalt und die didaktischen Methoden kritisch beurteilt.
