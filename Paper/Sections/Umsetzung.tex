\section{Umsetzung}
\subsection{Methoden zum Emotion Tracking}
Es gibt verschiedene Methoden um die Emotion eines Nutzers während dessen Interaktion mit einer Maschine zu tracken. Im folgenden werden beispielhaft verschiedene Methoden erläutert:

\subsubsection{Hautwiderstand und Hauttemperatur}
Das Paper "A Suggestion to Improve User-Friendliness Based
on Monitoring Computer User’s Emotions" beschreibt, wie Emotionen eines Nutzers durch dessen Hauttemperatur und Hautwiderstand bestimmt werden können. Die Autoren nutzen dafür ein Temperatur- und Hautwiderstandssensor, die mit einem Arduino verbunden sind. Die Daten der Sensoren werden in einer SQLite Datenbank gespeichert und auf einer Android App ausgegeben. Die Autoren stellten fest, dass eine Änderung der Hauttemperatur bzw. des Hautwiderstands  auf eine Emotionsänderung des Nutzers zurückzuführen ist \cite{EmotionTrackingGSR}.

\subsubsection{Blick Tracking}
Die Autoren des Papers "Improving Human-Computer Interaction
by Gaze Tracking" untersuchten, wie das Tracken des Blickes des Nutzers zur Steuerung von Maschinen verwendet werden kann. Unter anderem konnte festgestellt werden, wo und wie lange der Nutzer ein Objekt auf der Maschine betrachtet. Dabei wurde festgestellt, dass durch dieses Verfahren auch die Emotionen eines Nutzers bestimmt werden können. Beispielsweise verändert sich die Pupillengröße bei einer Emotionsänderung des Nutzers. Dabei nutzen die Autoren die integrierte Webcam in einem Laptop, um den Blick und somit die Emotionen eines Nutzers zu Tracken. Somit wird keine zusätzliche Hardware benötigt, wenn das Endgerät des Nutzers bereits eine Kamera integriert hat \cite{EmotionTrackingGaze}.

\subsubsection{Gesichtsausdrucks Tracking}
